\documentclass{article}
\usepackage{amsmath}
\usepackage{tcolorbox} % For creating custom-styled boxes
\usepackage{xcolor}

\begin{document}


\begin{equation}
\Huge {V} = {I}{R}
\end{equation}

\vspace{0.5cm} % Optional spacing for better layout
\begin{equation}
\Huge \boxed{\fcolorbox{black}{yellow}{$\Huge R = \frac{\rho L}{A}$}}
\end{equation}

\vspace{0.5cm} % Optional spacing for better layout
\begin{equation}
\huge \rho
\end {equation}

\vspace{0.5cm} % Optional spacing for better layout
\begin{equation}
   \Huge  {H} \cdot d{l} = NI \tag{1}
\end{equation}
\Large where
\vspace{0.5cm} % Optional spacing for better layout
\[
   \text{\Huge $dl = l_c$}
\]


\vspace{0.5cm} % Optional spacing for better layout
\begin{equation}
    \Huge {B} = \mu_0 \mu_r {H}\tag{2}
\end{equation}
\begin{align*}
\text{where} \quad & \mu_0 = 4 \pi \times 10^{-7} \quad \text{and} \quad \mu_r = 1.05 \quad \text{for air}. \\
\text{For steel,} \quad & \mu_r \text{ is in the range of } 2000 \text{ to } 6000.
\end{align*}



\vspace{0.5cm} % Optional spacing for better layout
\begin{equation}
   \Huge  \frac{B}{\mu_0 \mu_r} = {H}\tag{3}
\end{equation}

\vspace{0.5cm} % Optional spacing for better layout
\begin{equation}
   \Huge  \frac{B}{\mu_0 \mu_r} \cdot {l_c}= NI \tag{4}
\end{equation}


\vspace{0.5cm} % Optional spacing for better layout
\begin{equation}
   \Huge  {B} =\frac {\Phi}{A_c}\tag{5}
\end{equation}

\vspace{0.5cm} % Optional spacing for better layout
\begin{equation}
   \Huge  \frac{\Phi}{\mu_0 \mu_rA_c} \cdot {l_c}= NI \tag{6}
\end{equation}\vspace{0.5cm} % Optional spacing for better layout
where

\begin{tcolorbox}[colframe=black, boxrule=1mm, sharp corners=south] % Black box, 1mm line thickness, with sharp bottom corners
\[
   \Huge \frac{l_c}{\mu_0 \mu_r A_c} = R_\text{rel}
\]
\end{tcolorbox}

\vspace{0.5cm} % Optional spacing for better layout
\begin{equation}
   \Huge  {\Phi}R_\text{rel}= NI \tag{7}
\end{equation}\vspace{0.5cm} % Optional spacing for better layout

\vspace{0.5cm} % Optional spacing for better layout
\begin{equation}
   \Huge NI  = {\Phi}R_\text{rel}\tag{7}
\end{equation}\vspace{0.5cm} % Optional spacing for better layout

\begin{tcolorbox}[colframe=red, boxrule=1mm, sharp corners=south] % Black box, 1mm line thickness, with sharp bottom corners
\[
   \Huge NI= {\Phi}R_\text{rel}
\]
\end{tcolorbox}

\noindent
where 
\begin{align*}
NI &= \text{Magnetomotive Force (MMF)}, \\
\Phi &= \text{Flux}, \\
R_\text{rel} &= \text{Reluctance of the Magnetic Circuit}.
\end{align*}

\end{document}