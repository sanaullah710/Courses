\documentclass{article}
\usepackage{amsmath}
\usepackage{amssymb}
\usepackage{tikz}
\usetikzlibrary{shapes.geometric, arrows}
\usepackage{tcolorbox} % For creating custom-styled boxes
\begin{document}

\huge
\begin{itemize}
    \item \( l_g \) = length of the airgap
    \item \( l_c \) = length of the core
    \item \( A_g \) = Cross-sectional area of the airgap
    \item \( A_c \) = Cross-sectional area of the core
\end{itemize}

By using Ampere's law:

\begin{equation}
    \oint \mathbf{H} \cdot d\mathbf{l} = H_c l_c + H_g l_g = NI \tag{1}
\end{equation}





\begin{equation}
    B_c A_c = B_g A_g = \Phi_c \tag{2}
\end{equation}

\vspace{0.5cm} % Optional spacing for better layout
\begin{equation*}
   \Huge  {B} = \mu_0 \mu_r {H}
\end{equation*}

\begin{align*}
\text{where} \quad & \mu_0 = 4 \pi \times 10^{-7} \quad \text{and} \quad \mu_{rg} \approx 1\quad \text{for air}. \\
\text{For steel,} \quad & \mu_{rc} \text{ is in the range of } 2000 \text{ to } 6000.
\end{align*}

\vspace{0.5cm} % Optional spacing for better layout
\begin{equation*}
\begin{split}
   \Huge  {B} = \mu_0 \mu_{rc} {H}\\
  \Huge  {B} = \mu_0 \mu_{rg} {H}
\end{split}
\end{equation*}




\begin{equation*}
\begin{split}
\begin{aligned}
\text{Now,} \\
\mu_{rc} &= \text{Relative permeability of the core Material}\\
\mu_{rg} &= \text{Relative permeability of the Air}
\end{aligned}
\end{split}
\end{equation*}



\begin{equation}
    \mu_0 \mu_{rc} H_c A_c = \mu_0 \mu_{rg} H_g A_g = \Phi_c \tag{3}
\end{equation}

\begin{equation*}
    H_c = \frac{\Phi_c}{\mu_0\mu_{rc} A_c}, \quad H_g = \frac{\Phi_c}{\mu_0\mu_{rg} A_g}
\end{equation*}

Now, put \( H_c \) and \( H_g \) values in eq. (1):

\begin{equation}
    \frac{\Phi_c}{\mu_0 A_c} \left[ \frac{l_c}{\mu_{rc} A_c} + \frac{l_g}{\mu_{rg} A_g} \right] = NI
\end{equation}

\begin{align*}
\begin{split}
    \text{Now,} \quad A_C = A_g \\[0.5cm]\\
    \frac{l_g}{\mu_{rg}} &> >\frac{l_c}{\mu_{rc}}\\[0.5cm] \\
   \quad \text{Since } \mu_0 = 4\pi \times 10^{-7}\\[0.5cm] \\
    \text{while } \mu_{rc} &\approx 2000 \text{ to } 6000 \text{ for steel}\\[0.5cm] \\
    \text{So, } \quad \Phi_c = \frac{NI}{\frac{l_g}{\mu_0  \mu_{rg} A_g}}\\[0.5cm] \\
    \mu_{rg} \approx 1 \\[0.5cm]\\
    NI = \Phi_c \left[ \frac{l_g}{\mu_0 A_g} \right] \\ 
\end{split}
\end{align*}

\begin{tcolorbox}[colframe=red, boxrule=1mm, sharp corners=south] % Black box, 1mm line thickness, with sharp bottom corners
\[
    NI = \Phi_c \left[ \frac{l_g}{\mu_0 A_g} \right] \\ 
\]
\end{tcolorbox}

\end{document}
