\documentclass{article}
\usepackage{amsmath}
\usepackage{amssymb}
\usepackage{tikz}
\usetikzlibrary{shapes.geometric, arrows}
\begin{document}

\huge
\begin{itemize}
    \item \( l_g \) = length of the airgap
    \item \( l_c \) = length of the core
    \item \( A_g \) = Cross-sectional area of the airgap
    \item \( A_c \) = Cross-sectional area of the core
\end{itemize}

By using Ampere's law:

\begin{equation}
    \oint \mathbf{H} \cdot d\mathbf{l} = H_c l_c + H_g l_g = NI \tag{1}
\end{equation}





\begin{equation}
    B_c A_c = B_g A_g = \Phi_c \tag{2}
\end{equation}

\vspace{0.5cm} % Optional spacing for better layout
\begin{equation*}
   \Huge  {B} = \mu_0 \mu_r {H}
\end{equation*}

Now, 

\begin{equation}
    \mu_0 \mu_{rc} H_c A_c = \mu_0 \mu_{rg} H_g A_g = \Phi_c \tag{3}
\end{equation}

\begin{equation*}
    H_c = \frac{\Phi_c}{\mu_0\mu_{rc} A_c}, \quad H_g = \frac{\Phi_c}{\mu_0\mu_{rg} A_g}
\end{equation*}

Now, put \( H_c \) and \( H_g \) values in eq. (1):

\begin{equation}
    \frac{\Phi_c}{\mu_0 A_c} \left[ \frac{l_c}{\mu_{rc} A_c} + \frac{l_g}{\mu_{rg} A_g} \right] = NI
\end{equation}




\end{document}
